\chapter{Optimisation Performance}\label{ch:optimisation_performance}


\section{Comparison of Performed Reconstructions}
%Conclusions drawn from results
\noindent
From the loss curves in Figure \ref{fig:Loss_curve_optimal},
there is reason to believe that the Phantom reconstructions converged near the global minimum, while the carbon knot reconstructions
were trapped in a local minimum which had half the initial error.
Since the Phantom data set was a simulated data set, it was possible to ensure the initial conditions to be sufficiently good.
This condition was not possible to enforce upon the carbon knot data set, since it was experimental data.
Nevertheless, when comparing the quality of the Phantom reconstructions in Figure \ref{fig:phantom_reconstruction_3D}, \ref{fig:phantom_reconstruction_2D},
and \ref{fig:phantom_reconstruction_2D_angles} with the carbon knot reconstructions in Figure \ref{fig:carbon_knot_reconstruction_3D}, \ref{fig:carbon_knot_reconstruction_2D_coeffs},
and \ref{fig:carbon_knot_reconstruction_2D_angles}, there is reason to believe that the loss curves cannot be directly compared.
For instance, both the "SYM"- and "AD"- reconstructions of the carbon knot were seemingly more converged than the "EXPSIN"-reconstruction of the Phantom data set.
However, this qualitative observation is not reflected by the loss curves.
One possible explanation might be that the angles of the carbon knot were not fully optimised, which limited the loss reduction, even though the other coefficients were optimised.

The most pressing issue when evaluating the reconstructions is, however, the performance of the EXPSIN cost function.
The preferred orientation for many voxels of the Phantom data set had not changed significantly from the initial conditions.
A combination of a too weak anisotropic signal together with the exponential term in the cost function was possibly the root cause.
An assessment of computed gradient values revealed that the gradient with respect to the orientation was many orders of magnitude smaller than that of the coefficients.
Effectively, an orientation optimisation step becomes negligible small, and only the isotropic coefficient is optimised.
In contrast, all optimisation parameters are expected to have computed gradients of similar magnitude when the form factor is modelled by a linear combination of spherical harmonics.
Moreover, each spherical harmonic function is independent of the others, while the exponential term in EXPSIN appears to affect the intensity parameter $A$.
This dependency is revealed by an emerged pattern in Figure \ref{fig:phantom_reconstruction_2D}. The anisotropic coefficient $B$ is unchanged where the $A$-parameter is strongest. Likewise, $B$ has been overshooted where $A$ is undershooted.
The exponential term in the cost function may therefore appear to be a poor choice when the signal is only weakly anisotropic.
However, the result seemed to be much better with a stronger anisotropic signal, as seen in the carbon knot reconstructions of the preferred orientation in Figure \ref{fig:carbon_knot_reconstruction_2D_angles}.
As mentioned in Section \ref{sec:reconstruction_physical_carbon_knot},
the main features of the "SYM"- and "AD"-reconstructions were also present in the "EXPSIN"-reconstruction.
However, the spatial resolution of the reconstruction was poorer.
It is important to remember another important difference between the different algorithms.
SYM and AD both performed stepwise optimisation, while EXPSIN optimised all coefficients simultaneously.
This seems to favour the "SYM"- and "AD"-algorithms, since they were able to optimise the coefficients in a more targeted manner, resulting in a smaller loss as seen in Figure \ref{fig:Loss_curve_optimal}.
Therefore, it would be interesting to investigate stepwise optimisation in the "EXPSIN"-algorithm as well.




\section{Assessment of Methodology}
%Discussion of Methodology. Why is it better to use an AD framework? Why not?
\noindent
The AD framework proved essential when developing the "EXPSIN"-algorithm. Not only did it allow for a swift implementation.
It was also painless to alter the expression, for instance by including normalisation and applying an absolute value operator to the anisotropic coefficent.
Furthermore, other adaptations of the SAXSTT algorithm could be easily implemented. For instance, SAXSTT could have been expanded to reconstruct other symmetries than uniaxial.
This could require spherical harmonics functions with $m \neq 0$. Alternatively, another form factor expression would have to be developed.
Regardless, the gradient of these functions would not have to be derived tediously, and the resulting complex expression would not have to be implemented manually.

On the other hand, the AD framework includes a certain overhead and additional computational time. Moreover, debugging of the AD framework proved to be a challenge, as much of the computation is built-in.
Exact knowledge of all the operations performed by the backpropagation is therefore not possible.


\section{Improved Cost Functions}
As already mentioned, the AD framework could be used to extend the SAXSTT algorithm to other symmetries, for instance by using spherical harmonics with $m \neq 0$.
In regards to the existing cost functions, the "EXPSIN"-algorithm could possibly be improved by doing stepwise optimisation.
There is, however, a probability that the intensity term and the expontential term are too "entangled" to be optimised separately.
A solution could be to do a compromise between the exponential term and the spherical harmonics terms.
One possible limitation of the use of a linear combination of spherical harmonics is that one only gets discrete levels of anisotropy.
In this thesis, only four anisotropic spherical harmonics functions were used. If an even sharper anisotropy is desired, the number of optimisation parameters would increase drastically.
In contrast, the degree of anisotropy from EXPSIN is continious, and the number of optimisation parameters is constant.
Therefore, it would be interesting to create a linear combination of an isotropic intensity term and the EXPSIN form factor.
The resulting form factor would be similar to:
\begin{equation}
    \label{eq:form_factor_linear_combination}
    \bm{\widehat{R}}(\bm{r'}, \bm{q'}) \sim A_{0}^{2} + A_{1}^{2} \frac{\exp\left(\mp |B| \sin^2(\Theta) \right)} {\int_{0}^{\pi} \exp\left( \mp |B| \sin^{2}(x) \right) \sin(x) dx}
\end{equation}
\noindent
where $A_{0}$ and $A_{1}$ are the coefficients of the isotropic and anisotropic intensities, respectively. As before, $B$ determines the anisotropic shape of the form factor.
Note that the exponential term can replace any uniaxial spherical harmonics function.