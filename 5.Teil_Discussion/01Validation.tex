\chapter{Validation of Implementations} \label{ch:validation_discussion}

\section{Validity of SAXSTT}\label{sec:validity_saxstt}
The general validity of SAXSTT is as expected by a gradient-based maximum likelihood estimation performed on a complex problem with thousands of optimisation parameters.
The algorithm is able to give a fairly accurate orientation mapping of nanostructures
throughout a three-dimensional bulk sample with a spatial resolution limited by the brilliance, introduced in \eqref{eq:brilliance}, of the X-ray beam.
To elaborate, the reconstructed results of SYM in
Figure \ref{fig:coefficient_comparison_AD_SYM},
\ref{fig:orientation_comparison_AD_SYM}, and
\ref{fig:phantom_reconstruction_3D} were all very close to the simulated tomogram used to generate the data.
At the same time, these results were obtained using initial conditions close to the true values for the coefficients and angles.
However, Figure \ref{fig:carbon_knot_reconstruction_3D} shows a vector field in accordance with the expected results based on the known physics of the carbon knot.
As mentioned in Section \ref{sec:data_set_carbon_knot}, the carbon knot was expected to produce highly anisotropic scattering perpendicular to the axis of fibers.
The only change made to the initial conditions in this case was to assume negative $a2$ and $a6$ due to the mentioned equatorial fiber scattering.
With that being said, the reconstruction of the carbon knot did consist of outliers and deviation for individual voxels.
Usually, these outliers are handled by applied regularisation, which has not been done in this thesis, as it would interfere with the comparison of the SYM and AD implementations.
Anyway, this means that the spatial resolution is also limited by the applied regularisation and smoothing of the reconstructed vector field.
SAXSTT cannot necessarily produce a perfect reconstruction of each and every voxel in the sample individually,
but regularised domains of oriented reciprocal space maps are most likely correct.



% Therefore, SAXSTT is, as mentioned, more of a qualitative than quantitative characterisation technique. %RSD???
%RSD: Mention that reconstruction results are as expected by the physics of the carbon knot. 

%RSD: Check if this should be here or in computational performance. 

%RSD: Past tense or Present Tense


\section{AD SAXSTT}\label{sec:ad_validation}
Based on the reconstructions performed on simulated data in Section \ref{sec:reconstruction_parallel},
and on experimental data in Section \ref{sec:reconstruction_physical_carbon_knot},
it is evident that the AD implementation is valid, because of its similarity to the verified SYM implementation.
Qualitatively, the SYM and AD algorithms converged to the same solution, as seen in Figure \ref{fig:carbon_knot_reconstruction_3D}, \ref{fig:carbon_knot_reconstruction_2D_coeffs}, and \ref{fig:carbon_knot_reconstruction_2D_angles}.
However, there was naturally some quantitative difference, as expected by two iterative numerical optimisation algorithms.
As seen in the comparison of the initial gradients in Figure \ref{fig:gradient_comparison},
there was a small numerical deviation between the gradients calculated symbolically and automatically in the first iteration of the optimisation.
Therefore, the convergence trajectories were slightly different from one another. However, in the event of successful reconstructions,
both algorithms will have found the same global minimum.
The root cause of the mentioned deviation has not been investigated rigorously, since the resulting impact is negligible,
and a likely cause for most of the gradients, except perhaps the deviations between $15\%$ and $30\%$, is the numerical precision of the computer.
%RSD: With that being said... Systematic? Bug? dunno


\section{EXPSIN Modification}\label{sec:saxstt_validation}
From the discussion in Section \ref{sec:validity_saxstt} and \ref{sec:ad_validation}, the validity of the EXPSIN algorithm is also strengthened.
Because the only difference between the implementation of EXPSIN and AD is a different expression for the form factor,
one can assume that in the case that AD has been correctly implemented, EXPSIN has also been correctly implemented.
Consequently, the results of this algorithm can be used to evaluate the qualities and artefacts associated with \eqref{eq:final_exp_sin_squared} as a functional for SAXSTT.
Firstly, several voxels were trapped with their initial conditions when handling weakly anisotropic meridional scattering, as shown in Figure \ref{fig:phantom_reconstruction_2D} and \ref{fig:phantom_reconstruction_2D_angles}.
Secondly, the algorithm managed to reconstruct the main features of the carbon knot, but the final result was noisy with each feature covering a larger number of voxels than expected; evident in Figure \ref{fig:carbon_knot_reconstruction_2D_coeffs} and \ref{fig:carbon_knot_reconstruction_2D_angles}.
With an assumption that EXPSIN has been correctly implemented, the reason for, and possible solutions to, these artefacts will be further discussed in the following chapter.

%all three algorithms are able to minimise the error between the reconstructed scattering intensities and the analytical intensities.
%Hence, the resulting reconstructions are most likely qualitatively accurate descriptions of the orientation and degree of anisotropy within the sample.
%However,

% RSD: Make it one page?