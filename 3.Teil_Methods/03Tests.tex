
\chapter{Calculations}

\section{Gradients of Alternative Functional} %Functional is function of functions

% Validation even though not validated.
The validation of the alternative functional was conducted in stages with increasing complexity.
It was important to be able to contrast errors in the implementation from false gradient expressions.
Firstly, the gradient of a simple integer scattering model was calculated. The cost function of this model, with its gradient, is listed in Table \ref{tab:integer_model}.

% Is it actually fitting to keep this in a table? Or would equations suffice? Or necessary at all, or a part of the results?
% Possibly move to appendix.
\begin{table}[h]
    \centering
    \caption{}
    \label{tab:integer_model}
    \begin{tabular}{ l c }
        \hline
        \textbf{Cost Function}       & $\epsilon_{n} = \sum_{n', n''} ( (\sum_{n} A) - I_{n})^2 $           \\
        \textbf{Gradient Expression} & $\frac{\partial \epsilon_{n}}{\partial A} = 2 (\sum_{n} A - I_{n}) $ \\
        \hline
    \end{tabular}
\end{table}

The input data for the model in the test cases were random multidimensional arrays of the initial tomogram $A$ and the experimental intensity $I_{n}$.
The respective shapes were (2,2,2) and (3,2,2).
For each of the three 2D-projections, the forward and backward pass were performed, and the gradient expression was evaluated. %RSD: Code listing easier?
This procedure was identical to the initial testing of AD described in Section \ref{sec:proof_of_concept_AD}.

Next, the procedure was repeated using the cost function of the alternative functional, Equation \eqref{eq:exp_sin_squared}.
However, the gradients with respect to the parameters $A$ and $B$ were validated before the orientation parameters $\theta_{op}$ and $\varphi_{op}$ were handled.


Finally, the gradient with respect to the orientation parameters were tested.
For this test, the lab orientation was initially set to $\left( \alpha = 0, \beta = 0 \right)$, and the position coordinate was mostly kept at $\phi = 0$ to simplify the validation.
Since the behaviour of the $A$ and $B$ parameters were known, debugging of the final implementation was easier. %? 
In addition, the final parts of the symbolic gradient expression were derived by hand. The derivation is listed in Appendix \ref{app:gradient_derivation}.


% Derivation of expressions here or in appendix? Also, orientation symbolic expressions wrong, but unsure if I have done a mistake or if there was a mistake in things derived earlier. 


\section{Reconstruction of Periodic and Aligned Nanostructures Using Automatic Differentiation SAXSTT}

% RSD: The size was set to ... because.... Each voxel is identical, so the params are the same for each voxel. Nope, not here but in data sets. 
% Experimented a bit with starting conditions. 

The reconstruction script was run for the data sets described in Section \ref{sec:reconstruction_data_sets} using both symbolically and automatically calculated gradients.
The same starting conditions were applied to all runs.
Parameter \textbf{a} was set to $10^{-4}$, and the default orientation was set to $\left( \theta = \frac{\pi}{4}, \varphi = \frac{\pi}{4} \right)$.





\section{Comparison of Automatic and Symbolic Differentiation}

% All included in results should be mentioned here, But may be located in separate chapters in results?
% Postprocessing? But unsure what I planned here. 

% Also include the new cost function. 

First and foremost, the gradient calculations were evaluated by the accuracy of the reconstruction.

Secondly, the gradients from iteration 1 were compared. Here, the automatic gradients should be identical to the symbolic ones.

\section{Comparison of Cost Functions}% Perhaps. Test the 

A Comparison of the cost functions was mainly conducted on the experimental data set.
However, the cost function of the alternative functional was also tested on simulated data set,
as it proved easier to ensure that the initial conditions were sufficiently close to the global minimum.