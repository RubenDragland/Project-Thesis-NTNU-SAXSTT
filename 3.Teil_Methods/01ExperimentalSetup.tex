\chapter{Data Sets} % or Original set up and include 
\label{ch:reconstruction_data_sets}

Data analysis has been performed on one data set of experimental data and two data sets of simulated data.
This chapter describes the data sets used in the analysis.

\section{Carbon Knot from Synchrotron Measurement} % RSD: Maybe add a picture of the carbon knot?
\label{sec:data_set_carbon_knot}
The experimental data set was scattering data of a carbon fiber knot obtained from a synchrotron measurement \cite{PMID_30821257}.
According to the mentioned paper, the sample was prepared by tying a bundle of carbon fibers into a knot, and securing the ends of the fibers to metallic frames.
An optical image of the carbon knot was retrieved from \cite{PMID_30821257}, and is shown in Figure \ref{fig:carbon_knot_image}.
\begin{figure}[htbp]
    \centering
    \includegraphics[width=0.2\textwidth]{../XRD_CT/figures/optical_image_ck_2.png}
    \caption[Optical Image of Carbon Fiber Knot]{Optical image of a bundle of carbon fibers tied into a knot. The image was retrieved from \cite{PMID_30821257}.}
    \label{fig:carbon_knot_image}
\end{figure}
The long interface between the carbon fibers and air was the main source of scattering in the experiment, and resulted in highly anisotropic equatorial scattering between $0.021-0.049 \mathrm{nm}^{-1}$ \cite{PMID_30821257}.
Highly anisotropic scattering is ideal when validating implementations of SAXSTT, and the carbon knot was therefore chosen as the experimental data set.
Moreover, the carbon knot was rotated in both the azimuthal and polar directions, and 255 projections were sampled. Using a beam size of $30\mu\mathrm{m} \times 20\mu\mathrm{m}$.
46, 34, and 34 scanning points where needed in the y, x, and z directions, respectively.
Consequently, the reconstructions had dimensions $(y,x,z) = (46,34,34)$, where most of the voxels, except those located near the centre, consisted of air.
This relatively modest amount of data allowed the reconstructions to be performed on a laptop in an hour or less, considerably faster than larger data sets that may require days of computations on a powerful server.   % Need some info on the data set. 
Additionally, the data set included a correction term in the metadata to account for offset between the axis of rotation
and the centre of the sample. This data set will simply be referred to as the \emph{carbon knot}.


\section{Simulated Parallel Uniaxial Nanostructures} % Picture or visualisation?
\label{sec:reconstruction_data_sets_periodic_parallel_uniaxial_nanostructures}
A data set representing periodic and parallel uniaxial nanostructures was simulated using parts of the SAXSTT algorithm.
To elaborate, a set of determined spherical harmonics parameters were evaluated by the forward pass.
For each orientation, bilinear interpolation was performed to retrieve the projected intensity from a predetermined 3D model.
The resulting projection data was stored, and utilised as target values during testing of the different algorithms.
In this way, it was possible to measure and compare the accuracy of the reconstruction algorithms by comparing the reconstructed data to model which was used to generate the projection data.

In contrast to the carbon knot, this data set had no surrounding void, and no correction to the axis of rotation.
Instead, the 3D model consisted of "parallel uniaxial nanostructures" that covered the entire field of view.
In other words, all voxels had the same orientation and intensity coefficients.
From the individual reconstructions, only an inner threedimensional slice was retrieved for data analysis.
Therefore, edge artefacts due to bilinear interpolation and convolution with a Gaussian kernel were minimised,
and the quality of the gradient calculations could be assessed with minimal influence from other sources of error.

% The different data set sizes are listed in Table \ref{tab:periodic_data_sets_sizes} in Appendix \ref{app:appendixA}.
% The respective filters are also listed in Appendix \ref{app:appendixA}, as their exact values are not important.
% It is, however, important to remember that these filters are Gaussian, and filter 1 had the size $(3,3)$ while filter 3 had the size $(5,5)$. % RSD: Add appendix reference.

% For some of the mentioned data set sizes, different simulated orientations and coefficients were used.
% However, the relation between the coefficients were kept constant.
% The values of the coefficients are listed in Table \ref{tab:periodic_data_sets_coefficients} in Appendix \ref{app:appendixA}.


\section{Simulated Aligned Parallel Nanostructures with Void}
\label{sec:aligned_parallel_nanostructures_in_air}
Another data set was simulated using the SAXSTT forward pass. In contrast to the previously mentioned simulated data set,
this data set consisted of parallel nanostructures with surrounding void. Therefore, many of the voxels have zero intensity.
Additionally, the data set had alignment corrections between the axis of rotation and the centre of the sample, like the carbon knot.
Unlike the carbon knot, which is a result of fiber scattering where scattering occurs perpendicular to the orientation of the fibers,
this simulated data set represented scattering parallel to the orientation of the nanostructures.
This data set will be referred to as the \emph{Phantom} data set.


