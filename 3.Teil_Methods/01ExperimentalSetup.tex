\chapter{Data Sets} % or Original set up and include 
\label{sec:reconstruction_data_sets}

Data analysis has been performed on one data set of experimental data and some data sets of simulated data.
This chapter describes the data sets used in the analysis.

\section{Carbon Knot from Synchrotron Measurement}
Experimental projection data of a carbon knot was acquired.
The data has dimensions $(y,x,z) = (46,34,34)$, with most of the voxels, except those located in the centre, are air.  % Need some info on the data set.
Additionally, the data accounts for rotation of the sample through alignment. % RSD: Is this correct?


\section{Periodic and Aligned Uniaxial Nanostructures}
A data set representing periodic and aligned uniaxial nanostructures were simulated using the SAXSTT forward pass.
To elaborate, the desired spherical harmonics coefficients were evaluated by the forward pass.
The resulting projection data was stored, and utilised as experimental data in reconstructions.
In this way, it was possible to measure and compare the accuracy of the reconstruction algorithms.

Numerous such data sets were generated. Mostly the size of the data sets was varied. However, different convolutional filters were also applied to the projected data.
The goal was to find the suitable compromise between computation time and influence from the boundaries, and thereby the accuracy of the reconstruction.

The different data set sizes are listed in Table \ref{tab:periodic_data_sets_sizes}. The respective filters are listed in Appendix. % RSD: Add appendix reference.

\begin{table}[h]
    \centering
    \caption{}
    \label{tab:periodic_data_sets_sizes}
    \begin{tabular}{ l c c c }
        \hline
        \textbf{Size} & \textbf{Voxels Evaluated} & \textbf{Offset} & \textbf{Filter} \\
        (7,7,7)       & $3^{3}$                   & 4               & 1               \\
        (15,15,15)    & $11^{3}$                  & 4               & 1               \\
        (25,25,25)    & $17^{3}$                  & 8               & 1               \\
        (35,35,35)    & $27^{3}$                  & 8               & 1               \\
        (35,35,35)    & $17^{3}$                  & 18              & 3               \\
        \hline
    \end{tabular}
\end{table}

For some of the mentioned data set sizes, different simulated orientations and coefficients were used.
However, the relation between the coefficients were kept constant.
The values of the coefficients are listed in Table \ref{tab:periodic_data_sets_coefficients}.

\begin{table}[h]
    \centering
    \caption{}
    \label{tab:periodic_data_sets_coefficients}
    \begin{tabular}{ c c c }
        \hline
        \textbf{a} & $\bm{\theta}$   & $\bm{\varphi}$   \\
        0.69       & $\frac{\pi}{3}$ & $\frac{2\pi}{3}$ \\
        16.9       & $\frac{\pi}{3}$ & $\frac{2\pi}{3}$ \\
        \hline
    \end{tabular}

    \begin{tabular}{ c c c c c c}
        \textbf{a0} & \textbf{a2}    & \textbf{a4}    & \textbf{a6}    & $\bm{\theta}$ & $\bm{\varphi}$ \\
        1 a         & $\frac{1}{3}$a & $\frac{1}{6}$a & $\frac{1}{12}$ & $\theta$      & $\varphi$      \\
        \hline
    \end{tabular}
\end{table}


