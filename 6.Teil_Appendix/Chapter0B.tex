% Appendix X

\chapter{Appendix}\label{app:appendixB}

%----------------------------------------------------------------------------------------

% Content begins here

All code can be found, in its raw and unedited form, on the GitHub repository \url{https://github.com/RubenDragland/XRD_CT}.
This appendix is a short description of the different folders and files in the repository.
The necessary packages and libraries for the Python scripts can be found in the \emph{SAXSTT\_spec.txt}-file.


\section{Reference to PSI Packages}

The original scripts for SAXSTT was downloaded with permission from the PSI website, \url{http://www.psi.ch} \cite{liebi2015nanostructure}.
These are located in $\emph{3DSAXS\_package}$, \newline $\emph{cSAXS\_matlab\_base\_package}$ and $\emph{cSAXS\_matlab\_sSAXS}$.
Together with the original files are modified versions of SAXSTT components written in MATLAB.
Most essentially are the $\emph{Run}$-files, which are used to initilise the scripts efficiently.

\section{Pytorch Automatic Differentiation}
Within the folder $\emph{Autodiff\_package}$, intitial tests of autograd engines can be found.
Additionally, gradient calculation for AD SAXSTT can be found, both optimised and unoptimised versions.
The folder and the files have not been cleaned at this current date, but might eventually be cleaned and updated with additional functionality.

\section{Data Analysis}
As for the data, the folder $\emph{Data sets}$ contains some of the data used in the thesis.
For instance, the carbon knot data set can be found here.
$\emph{Figures}$ contains the editable svg-files to the figures that were created in $\emph{Inkscape}$.
For figures created using $\emph{Matplotlib}$, please see the $\emph{Plotting}$-folder.
Raw results and performed data analysis can be found in the $\emph{Results}$ folder.



