% Abstract

%\renewcommand{\abstractname}{Abstract} % Uncomment to change the name of the abstract

\pdfbookmark[1]{Abstract}{Abstract} % Bookmark name visible in a PDF viewer

\begingroup
\let\clearpage\relax
\let\cleardoublepage\relax
\let\cleardoublepage\relax

\chapter*{Abstract}
% Short summary of the contents\dots a great guide by 
% Kent Beck how to write good abstracts can be found here:  
% \begin{center}
% \url{https://plg.uwaterloo.ca/~migod/research/beckOOPSLA.html}
% \end{center}

Small Angle X-ray Scattering Tensor Tomography is a form of computed tomography
which aims at using Small Angle X-ray Scattering (SAXS) data to do micrometer resolution orientation mapping
of uniaxial nanostructures in three-dimensional amorphous samples.
The reconstruction algorithm is similar to a maximum likelihood estimation that utilises gradient descent to find the optimal 3D vector field
to describe the orientation of nanostructures throughout the bulk sample.
In this thesis, the gradient calculation of the SAXSTT algorithm is implemented using automatic differentiation.
Thereby, an additional dimension of versatility has been added to world of SAXSTT, and the technique can easily and swiftly be optimised further and expanded.
Revealed weaknesses of the algorithm include bilinear interpolation artefacts,
and the fact that the deterministic conjugate gradient descent algorithm converge to local minima without sufficiently good initial conditions.
Proposed solutions to the mentioned weaknesses included implementation of Fourier-Sinc interpolation and experimentation with second order methods or the stochastic gradient descent algorithms, such as ADAM or AMSGrad.
The proposed form factor function,
\begin{equation*}
    \bm{\widehat{R}}(\bm{r'}, \bm{q'}) \sim A_{0}^{2} + A_{1}^{2}\frac{ \exp\left(\mp |B| \sin^2(\Theta) \right) } {\int_{0}^{\pi} \exp\left( \mp |B| \sin^{2}(x) \right) \sin(x) dx},
\end{equation*}
where the sign in the exponent depends on the scattering model, was estimated to expand the functionality of the algorithm in the case of uniaxial nanostructures.
Similarly, Spherical Harmonics functions with $m \neq 0$ can be implemented due to automatic differentiation,
and provides the ability to do orientation mapping of other anisotropic symmetries.

\clearpage
\chapter*{Zusammenfassung}
Spitzer Winkel Röntgen Streuung Tensor Tomographie ist eine Computertomographietechnik,
die die Orientierungen der uniaxialen Nanostrukturen in dreidimensionalen amorphen Materialen mit mikrometer Auflösung rekonstruieren.
%%%

\endgroup

\vfill

