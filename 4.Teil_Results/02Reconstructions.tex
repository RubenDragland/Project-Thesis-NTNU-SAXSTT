\chapter{Performed SAXSTT Reconstructions}

\section{Reconstruction of Simulated Parallel Nanostructures}
\label{sec:reconstruction_parallel}

With a chosen data set size,
the SAXSTT algorithm was performed on the dataset of parallel nanostructures, as mentioned in Section \ref{sec:pp_nanostructures_reconstruction}.
Firstly, the accuracy of automatically calculated gradients compared to symbolically calculated gradients was analysed.
Figure \ref{fig:coefficient_comparison_AD_SYM} and \ref{fig:orientation_comparison_AD_SYM} show distributions of spherical harmonics parameters for converged SAXSTT reconstructions.
In addition to the raw data, the results include curve fitting of the distributions to Lorentzian functions. % The optimal parameters are listed in table...
Initially, a Gaussian curve fitting was performed, but the Lorentzian function proved to be a better fit.
The peak and full width at half maximum (FWHM) of the Lorentzians are listed in Table \ref{tab:curve_fitting}.

\begin{figure}[h!]
    \centering
    \includesvg[width=1\textwidth]{../XRD_CT/Plotting/thesis_plots/SH_coeffs_0align_closecoeffs.svg}
    \caption{ The coefficients converge to a Lorentzian distribution
        around the true value when the initial conditions have the correct sign.  }
    \label{fig:coefficient_comparison_AD_SYM_difficult}
\end{figure}

\begin{figure}[h!]
    \centering
    \includesvg[width = 1\textwidth]{../XRD_CT/Plotting/thesis_plots/SH_orientation_0align_closeangles.svg}
    \caption{  The orientation parameters converge to a Lorentzian distribution
        around the true value when the initial conditions are sufficiently close to the true value.}
    \label{fig:orientation_comparison_AD_SYM_difficult}
\end{figure}


\begin{table}
    \centering
    \caption{  The peak and FWHM of the Lorentzian curve fits
        for the coefficients and orientation parameters for the reconstruction
        with good initial conditions.}
    \label{tab:curve_fitting}
    \begin{tabular}{ c c c c c c }
        \hline %\toprule
        \textbf{}      &                   & \multicolumn{2}{c}{\textbf{Automatic (AD)}} & \multicolumn{2}{c}{\textbf{Symbolic (SYM)}}                                             \\
        \textbf{}      & \textbf{Solution} & \textbf{Peak}                               & \textbf{FWHM [\num{e-3}]}                   & \textbf{Peak} & \textbf{FWHM [\num{e-3}]} \\
        \hline %\midrule
        \textbf{a0}    & 0.690             & 0.691                                       & 5.834                                       & 0.691         & 6.764                     \\
        \textbf{a2}    & 0.230             & 0.230                                       & 2.930                                       & 0.229         & 3.906                     \\
        \textbf{a4}    & 0.115             & 0.000                                       & 0.000                                       & 0.000         & 0.000                     \\
        \textbf{a6}    & 0.575             & 0.000                                       & 0.000                                       & 0.000         & 0.000                     \\
        $\bm{\theta}$  & 1.047             & 0.000                                       & 0.000                                       & 0.000                                     \\
        $\bm{\varphi}$ & 1.047             & 0.000                                       & 0.000                                       & 0.000                                     \\
        \hline %\bottomrule
    \end{tabular}
\end{table}

However, a more important question is to assess the quality of the algorithms when the initial conditions are not close to the optimal solution.
Figure \ref{fig:coefficient_comparison_AD_SYM_difficult} and \ref{fig:orientation_comparison_AD_SYM_difficult} show the distributions of spherical harmonics parameters for reconstructions
where the parallel nanostructures are rotated by 60 degrees relative to the case in Figure \ref{fig:coefficient_comparison_AD_SYM} and \ref{fig:orientation_comparison_AD_SYM}.
%The results show that the algorithm is able to find the correct solution, even when the initial conditions are far from the optimal solution.
% RSD: describe the convergence.

\begin{figure}[h!]
    \centering
    %\includesvg[height = \textheight]{../XRD_CT/Plotting/thesis_plots/SH_coeffs_vertical_distributioncoeffs.svg}
    \caption{f}
    \label{fig:coefficient_comparison_AD_SYM}
\end{figure}

\begin{figure}[h!]
    \centering
    %\includesvg{../XRD_CT/Figures/Flowchart_optimisation_algorithm.svg}
    \caption{ f   }
    \label{fig:orientation_comparison_AD_SYM}
\end{figure}


%RSD: Change this a bit. Use the phantom data set, 
As a final note, the reconstruction of the parallel nanostructures, with favorable orientation, was performed using the alternative functional.
In this case, it is not possible to directly compare coefficients, except for those controlling orientation.
Thus, the FWHM of the parameter distribution turned out to be an important metric.
The same assessment as conducted in Figure \ref{fig:coefficient_comparison_AD_SYM} and \ref{fig:orientation_comparison_AD_SYM} was performed for the parameters of the alternative functional.
The results are shown in Figure \ref{fig:coefficient_comparison_AD_SYM_alternative} and \ref{fig:orientation_comparison_AD_SYM_alternative}.

%RSD: Believe paragraph above will be abolished. Instead, use 3D representation of reconstruction on phantom data set.
In addition, visual representation of the reconstructed nanostructures is emphasised. %? % Make delicious plot. 
However, a degree of anisotropy was also derived and used as a metric in this case. % RSD: Do this. 
The average degree of anisotropy was calculated for the solution and each of the reconstructions, and can be found in Table \ref{tab:anisotropy}.
To elaborate, deviation in terms of degree of anisotropy for the different reconstructions are shown in Figure \ref{}, \ref{}, and \ref{} for symbolic, automatic, and alternative gradients, respectively.

%RSD: Include phantom cube data here. Slices or quiver? Quiver if possible. 

\clearpage
\section{Reconstruction of Physical Carbon Knot}

% How to visualise this? Degree of Anisotropy?
% Slices. center-most
% 3D Quiver plot. 

%The general reconstruction ability of the different gradient computation methods 
The different methods were compared using actual synchrotron data of a carbon fiber knot.
Figure \ref{} shows the centre slice (z = 17) of the isotropic coefficent, a0, reconstructed using symbolically, automatically, and alternatively calculated gradients, respectively.
Note that the a0 coefficient corresponds to coefficient A in the alternative cost function \eqref{eq:exp_sin_squared}.
\begin{figure}[h!]
    \centering
    %\includesvg[height = \textheight]{../XRD_CT/Plotting/thesis_plots/SH_coeffs_aligned_distributioncoeffs.svg}
    \caption{  Include slice of carbon knot, a0 }
    \label{}
\end{figure}
Furthermore, \ref{} shows the same slice, but the orientation, theta, is visualised.
\begin{figure}[h!]
    \centering
    %\includesvg[height = \textheight]{../XRD_CT/Plotting/thesis_plots/SH_coeffs_aligned_distributioncoeffs.svg}
    \caption{  Include slice of carbon knot, theta }
    \label{}
\end{figure}

For the remaining coefficients, the degree of anisotropy was calculated and used as a metric to compare the different methods.
% Moreover, a three-dimensional quiver plots combine all reconstructed parameter,
% and can better compare the reconstruction using spherical harmonics function to the reconstruction using the alternative functional.

\begin{figure}[h!]
    \centering
    %\includesvg[height = \textheight]{../XRD_CT/Plotting/thesis_plots/SH_coeffs_aligned_distributioncoeffs.svg}
    \caption{  Include slice of carbon knot, theta }
    \label{}
\end{figure}

