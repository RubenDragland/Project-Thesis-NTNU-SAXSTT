\chapter{Computational Performance}



\section{Time Complexity of Gradient Computation}

Figure \ref{fig:gradient_time_complexity} shows the time complexity of the gradient computation for AD and SYM.
The dotted line indicates a time complexity of $\mathcal{O}(N)$, where $N$ is the number of voxels.
All tests were approximately parallel to the dotted line, which indicates that the time complexity of both implementations is indeed $\mathcal{O}(N)$.
Note that \emph{AD laptop} is tested on another device than the rest.
Regarding \emph{AD CPU/GPU}, the CPU and GPU automatic gradient calculations are merged, since the performance did not change as a result of choice of device.% as the result was the same.
Moreover, it is important to notice that the speed-up of the parallelisation is not proportional to the number of cores, which was 12.
%RSD: Update with SYM as well? Manually. Not the most important thing. 
%RSD: Mentioned Laptop vs Server. Note that GPU and CPU are equally fast. However, if algorithm extended so that the amount of transfer of data is at a relative minimum, a significant speed-up will occur. 
\begin{figure}[h!]
    \centering
    \includesvg[width=1\textwidth]{../XRD_CT/Plotting/thesis_plots/gradient_computation.svg}
    \caption{ The gradient computation time of AD and SYM for different number of voxels using a laptop, GPU, MEX-functions, and parallelisation, respectively. % Should test CPU on Gamma instead. Mind myself. Not the most important thing. Change text 
        The time complexity was about $\mathcal{O}(N)$ for both methods, where N is the number of voxels.}
    \label{fig:gradient_time_complexity}
\end{figure}

\clearpage
\section{Convergence of SAXSTT}

Figure \ref{fig:Loss_curve_optimal} shows the convergence of SAXSTT after 100 iterations for the different implementations.
Specifically, reconstructions of the carbon knot and the Phantom data set are included.
The characteristic convergence of the Spherical Harmonics method is evident.
Since the different sets of parameters are optimised independently before the final optimisation, several exponential decays are visible.
The same step-wise convergence was not implemented for EXPSIN, which therefore experiences an initial exponential decay followed by an almost flat convergence.
In fact, only $10\%$ of the EXPSIN iterations showed exponential decay; the rest were almost flat.
Another fascinating observation was that the loss of the stepwise optimisation drastically increased after 30 iterations, meaning the first iteration of the anisotropic coefficients optimisation.
Moreover, the loss of the carbon knot reconstruction was only halved after 100 iterations, whereas the loss of the Phantom data set was reduced by a factor of 10.
In terms of computational time, the symbolic reconstruction of the Phantom data set was significantly faster than the rest, since only 50 of the 100 iterations were performed.
Apart from that, the "AD"-algorithm and the "EXPSIN"-algorithm are within twice the time of the "SYM"-algorithm.


%Update with EXPSIN as well? Evt for carbon knot. Evt include all runs or do normalisation. Difficult to know right now. 

\begin{figure}[h!]
    \centering
    \includesvg[width=1\textwidth]{../XRD_CT/Plotting/thesis_plots/Loss_curve_optimal.svg} %RSD: Remember inkscape formatter and scaling. More changes needed to the plot. Generalise number og lines. 
    \caption{ The loss curves of the respective performed reconstructions. Each curve is normalised to the initial loss.
        CK is short for carbon knot, and P stands for Phantom. }
    \label{fig:Loss_curve_optimal}
\end{figure}