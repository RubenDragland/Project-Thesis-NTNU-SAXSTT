\chapter{Validation of Gradients}

% RSD: Make some sections about the results of the run. 

\section{Symbolic Gradients of Alternative Functional}

The derived symbolic gradient expression of the alternative functional can be found in Appendix \ref{app:appendixB}.

As mentioned in Section \ref{sec:calc_alt_functional}, the derived expression was compared to automatically calculated gradients for random values.
The symbolic gradient expressions for coefficients $A$ and $B$ agreed with the automatically calculated gradients for all sampled initial values.
The gradients with respect to the orientation parameters $\theta_{op}$ and $\varphi_{op}$, however, were incorrect.
Even though two different methods were used to derive the latter gradients, none proved to be correct.

%Conclude here?
As a result, the gradients of the alternative functional were correctly derived for the coefficients $A$ and $B$,
but wrongly derived for the orientation parameters $\theta_{op}$ and $\varphi_{op}$.
Furthermore, SAXSTT using the alternative functional had to be implemented using automatically calculated gradients.

\section{Automatic Gradients of SAXSTT}

It is worth comparing the gradients of the first iteration of the SAXSTT algorithm implemented using symbolically and automatically calculated gradients, respectively.
Figure \ref{fig:gradient_comparison} shows
the relative deviation of the automatically calculated gradients with respect to the symbolically calculated gradients for the first iteration of the $a0$-optimisation.
This parameter was chosen for the comparison, because it was the most easily retrievable data.
A tendency shown in the scatter plot is that the relative deviation was larger for smaller gradients.
Moreover, the positive deviation indicates that the symbolically calculated gradients are generally larger in terms of magnitude.
The colours in Figure \ref{fig:gradient_comparison} indicate the location of the voxel. A high radius means a voxel located close to the edge of the sample.
%since both gradients were negative, it means that the symbolically calculated gradients were larger than the automatically calculated gradients. % RSD: Check this 
% RSD: Think scatter plot is nice here

\begin{figure}[h!]
    \centering
    % RSD: Vector graphics slow due to number of elements. png might be better here. But need higher resolution.
    %\includesvg[width= 1\textwidth]{../XRD_CT/Plotting/thesis_plots/SH_diff_grads.svg}
    \includegraphics[width=1\textwidth]{../XRD_CT/Plotting/thesis_plots/SH_a0_gradients_raster.png}
    \caption{The deviation data is sorted by symbolic gradient value.
        The majority of the voxels have a deviation below $5\%$, while a subsection of the data has a deviation between $15\%$ and $30\%$.
        Naturally, the relative deviation is larger for smaller gradients.
        Also, voxels located close to the edge of the sample have a larger deviation.}
    \label{fig:gradient_comparison}
\end{figure} % Color value according to index radius. 

% %RSD: Worth mentioning when radius expressed?
% Another way to compare the gradients is to look at the location of the deviation.
% Figure \ref{fig:gradient_comparison_spatial} shows the calculated relative gradient deviation for slices close to the edge and the centre of the tomogram.

% \begin{figure}[h!]
%     \centering
%     %\includesvg{../XRD_CT/Figures/Flowchart_optimisation_algorithm.svg}
%     \caption{
%     }
%     \label{fig:gradient_comparison_spatial}
% \end{figure}

