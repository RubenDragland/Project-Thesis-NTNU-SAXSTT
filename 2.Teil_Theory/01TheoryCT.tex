
\chapter{Computed Tomography}

%RSD: Should do more precise calculations?
\section{X-rays}\label{sec:CT_Xrays}

X-rays are electromagnetic waves with energy in the orders of $keV$. Planck's Equation,
\begin{equation}\label{eq:Plancks_eq}
    E = \hslash \omega = 2\pi \hslash \frac{c}{\lambda},
\end{equation}
\noindent
relates the energy of a photon $E$ to the angular frequency $\omega$ or wavelength $\lambda$ of the corresponding electromagnetic wave.
$c = 2.99792 \times 10^8 m/s$ is the speed of light.
The other constant is the reduced Planck's constant $\hslash = 1.0543 \times 10^{-34} Js$  \cite{blokhin1961physics}.
Whence, X-rays typically have wavelengths in the sub-nanometer range.


Excitation and acceleration are the most common phenomena where X-ray radiation occurs.
X-ray radiation from excitation, called "Characteristic X-ray radiation", occurs when a highly energetic electron collides into a target atom.
The accelerated electron transfers enough energy to eject an inner-shell electron with energy $E_i$,
and an outer electron with energy $E_f$ lowers its energy-state by filling the vacancy,

\begin{equation}\label{eq:Char_radiation}
    E_{\mathrm{photon}} = - \Delta E = - (E_f - E_i).
\end{equation}

Due to conservation of energy, this process causes emission of a photon, as described in Equation \eqref{eq:Char_radiation}.
As the atomic energy levels are discrete, this process gives rise to a spectrum of discrete X-ray emission lines.
%\cite{Stark, G.. "X-ray." Encyclopedia Britannica, January 30, 2020. https://www.britannica.com/science/X-ray.}.


Additionally, scattering events occur when electrons pass through an anode material.
These events accelerate the electrons in new directions, and X-rays known as $"Bremsstrahluhng"$ are emitted. % Include Intensity relation of Bremsstrahlung?

%Improve, Figure or equations
The synchrotron is the last common X-ray source.
Generally, charged particles are accelerated to very high velocities, and because magnets exert forces on charged particles, they accelerate in accordance with
Lorentz force and Newton's second law of motion,

\begin{equation}\label{eq:Lorentz_force}
    m\vectorBU{a} = q \vectorBU{v} \times \vectorBU{B}.
\end{equation}
As described, the acceleration will be parallel to the cross product between the velocity $\vectorBU{v}$ and the external magnetic field $\vectorBU{B}$ \cite{crater1994general}. %RSD: Cite
%\cite{Britannica, T. Editors of Encyclopaedia. "synchrotron." Encyclopedia Britannica, February 7, 2018. https://www.britannica.com/technology/synchrotron.}
% Add equation?


\section{Beer-Lambert's Law}
The X-ray beam intensity attenuates upon interacting with matter due to photoelectric absorption, elastic Rayleigh scattering, and inelastic Compton scattering.
The attenuation coefficient $\mu$ describes this attenuation in an inhomogeneous sample as

\begin{equation}
    I(s) = I(0) \exp(- \int_{0}^{s} \mu(\nu) d\nu),
\end{equation}
\noindent
where $s$ is the thickness of the sample, and $I(0)$ is the initial intensity. The spectral dependence $\mu(\nu)$ is here neglected, as it often is, as the beam is assumed to be almost fully monochromatic \cite{buzug2009computed}.
A simple manipulation of the expression gives the projection line integral

\begin{equation}\label{eq:projection}
    p(s) = -\ln(\frac{I(s)}{I(0)} ) = \int_{0}^{s} \mu(\nu) d\nu.
\end{equation}

%RSD: Rewrite. 
\section{Radon Transform}
The projection line integral in Equation \eqref{eq:projection} is closely related to the Radon transform of an object function $f(x,y)$ for a single orientation $\theta$ \cite{zeng2010medical}.
Confidence in this statement may be achieved by comparing Equation \eqref{eq:projection} with a single-angle Radon transform,

\begin{equation}\label{eq:Radon_transform}
    p_{\theta}(r) = \int_{-\infty}^{\infty} f(r,\nu) d\nu.
\end{equation}
Collected Radon transform data is called a sinogram, and is used to reconstruct the original object function $f(x,y)$ \cite{gonzalez2018digital}.
%The Radon transform of a function $f(x,y)$ along the angle $\theta$ is defined in Equation \eqref{eq:Radon_transform} purely as a line integral along the desired angle. %?

%\section{Detectors}??? Cannot be relevant

\section{Fourier Slice Theorem}
%\section{Projections}
The key in computed tomography is to determine the spatial dependency of the attenuation coefficient $\mu(\bm{r})$.
By sampling many projections, meaning line integrals from different orientations, data necessary to reconstruct a three-dimensional image is collected.
For a given cross section of the sample $f(x, y)$, the detected intensity is plotted as a function of projection number and detector number in what is called a sinogram.
By utilising this sinogram and the Fourier slice theorem, the object $f(x, y)$ may be determined by other means than computing the full inverse Radon transform.
%\section{Fourier Slice Theorem}

The Fourier slice theorem states that the Fourier transform of a projection
is a slice of the 2D Fourier Transform of the region from which the projection was obtained \cite{gonzalez2018digital}.
Consequently, the full 2D Fourier transform
$F(\omega_x, \omega_y)$ of an object $f(x,y)$ can be constructed from a series of 1D Fourier transforms
$P(\omega)$ of projections $p(s)$ with different orientations \cite{zeng2010medical}.
In other words, the Fourier transform of a single projection makes up one angle of the full 2D Fourier transform,

\begin{equation}\label{eq:Fourier_slice}
    P(\theta, \omega) = F \left( \omega \cos \theta, -\omega \sin \theta \right). %\int_{-\infty}^{\infty} P(\omega) e^{i \omega_x x} dx.
\end{equation}
%  that the 1D Fourier transform $P(\omega)$ of the projection $p(s)$ of the object $f(x,y)$ is equal to a slice through the origin of the 2D Fourier transform $F(\omega_x, \omega_y)$ \cite{}. 


% ILLUSTRATE with a Figure
%See and reconstruct Figure 5.13 in buzug2009computed possibly... or the one in zeng2010medical


\section{Filtered Back Projection}
In short, the filtered back projection (FBP) algorithm reconstructs the object by forward and inverse Fourier transforms.
Firstly, a sinogram of projections is mapped to frequency space in polar coordinates by subsequent 1D Fourier transforms, mathematically described as:

\begin{equation}\label{eq:FBP_1}
    P(\theta, \omega) = \int_{-\infty}^{\infty} p(\theta,r)e^{-2\pi i\omega r} dr.
\end{equation}

With this the 2D Fourier transform $F(u,v)$ of the object $f(x,y)$ is found.
The final step is an inverse 2D Fourier transform with, for instance, a $ramp$-filter of $|\omega|$ to account for the radial distribution of points in polar coordinates.
This filter is also the Jacobian of the area integration element in the polar Fourier space.
Consequently, the object function can be expressed as

\begin{equation}
    f(x,y) = \int_{0}^{\pi} \int_{-\infty}^{\infty} \left|\omega\right| P(\theta,\omega)e^{-2\pi i\omega (x\cos\theta - y\sin\theta)} d\omega d\theta.
\end{equation}

%RSD: Read through a couple of times. 






