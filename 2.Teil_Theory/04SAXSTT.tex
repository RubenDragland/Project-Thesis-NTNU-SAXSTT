\chapter{Small Angle X-ray Scattering Tensor Tomography}

\section{X-ray Pencil Beam} % Or in experimental setup. Define brilliance
The first crucial component of SAXSTT is the quality of the X-ray beam.
Obviously, more photons increase the signal-to-noise ratio, and thus the resolution of the electron density distribution is improved.
Moreover, the voxel size is determined by the size of the X-ray beam, where a focused beam results in higher resolution of reconstructed tensor at the cost of reconstruction time.
Finally, the most consistent scattering occurs when the X-ray beam is almost purely monochromatic, meaning that the spectral bandwidth is narrow.
All these properties are defined by the brilliance of the X-ray beam, given as
\begin{equation}
    Brilliance = \frac{Photons/second}{\left( mmrad \right)^{2} \left( mm^{2} \right) \left( 0.1\% BW \right).},
\end{equation}
where $BW$ is a commonly used abbreviation for spectral bandwidth \cite{mcmorrow2011elements}.

\section{Experimental Setup}

\section{Modelling of Anisotropic Scattering}

\section{Optimisation Algorithm}