\chapter{Small Angle X-ray Scattering Tensor Tomography}
\label{chap:SAXSTT}

Small Angle X-ray Scattering Tensor Tomography is a characterisation technique where the aim is to reconstruct the reciprocal space map in each voxel of a three-dimensional sample
that holds nanostructures with anisotropic electron distribution.
In this way, one can map the orientation and the anisotropy of the nanostructures throughout the three-dimensional bulk sample.

\section{X-ray Pencil Beam} % Or in experimental setup. Define brilliance
The first crucial component of SAXSTT is the quality of the X-ray beam.
Obviously, more photons increase the signal-to-noise ratio, and thus the resolution of the electron density distribution is improved.
Moreover, the voxel size is determined by the size of the X-ray beam, where a focused beam results in higher resolution of the reconstructed tensors at the cost of reconstruction time.
Finally, the most consistent scattering occurs when the X-ray beam is almost purely monochromatic, meaning that the spectral bandwidth is narrow.
All these properties are defined by the brilliance of the beam, given as
\begin{equation}
    Brilliance = \frac{Photons/second}{\left( mmrad \right)^{2} \left( mm^{2} \right) \left( 0.1\% BW \right).},
\end{equation}
where $BW$ is a commonly used abbreviation for spectral bandwidth \cite{mcmorrow2011elements}.

\section{Experimental Setup}
The experimental part of the SAXSTT process consists of collecting a series of two-dimensional SAXS patterns from different orientations $\left(\alpha,\beta\right)$.
Figure \ref{fig:orientations} shows the definition of the angles $\alpha$ and $\beta$. % Fix. 
In order to reconstruct tensors, the sample must be rotated around two axes of rotation, unlike conventional CT which only requires one \cite{liebi2018small}.
The synchrotron beam scans across the sample in a raster pattern for each orientation, and collects the combined scattering intensity for each sector of the diffractogram, respectively.
Figure \ref{fig:experimental_setup} shows the experimental setup of a given SAXSTT experiment for a single orientation and at a single scanning point $(\alpha, \beta,x,y)$.

\begin{figure}
    \centering
    \includesvg[width=1\textwidth]{Figures/SAXSTT_test.svg}
    \caption{Experimental setup of a SAXSTT experiment. The beam canon represents the synchrotron beam. The cube represents a single row of voxels in the sample.
        The scattering intensity from eight sectors of the diffractogram is collected for each scanning point $(\alpha, \beta,x,y)$.
        Due to uniaxial symmetry, this actually represents a partitioning of the diffractogram into 16 sectors.}
    \label{fig:experimental_setup}
\end{figure}

%Figure of scanning orientations?


\section{Modelling of Anisotropic Scattering}
The coordinate systems used in this Thesis follow the convention of the article "Small-angle X-ray scattering tensor tomography:
model of the three-dimensional reciprocal-space
map, reconstruction algorithm and angular
sampling requirements" \cite{liebi2018small}.
The reciprocal space map of a voxel is denoted $\bm{\widehat{R}}(\bm{r'}, \bm{q'})$,
where $\bm{r'}$ and $\bm{q'}$ are the position and reciprocal vector in the object-coordinate system, respectively.
Transformation from the lab-coordinate system $\bm{a}$ to the object-coordinate system $\bm{a'}$ for a given property $\bm{a}$
is given by the applying the rotation matrix $\bm{R}_{n}^{exp}$ of the given projection $n$ with orientation $\left(\alpha,\beta\right)$.

% Fig to show alpha and beta. Thinking simple sphere with dots. 

One choice of representing the reciprocal space map is to model the form factor as a linear combination of spherical harmonics functions.
The intensity of the reciprocal space map is then,
as given in Equation \eqref{eq:scattering_intensity_classical}:

\begin{equation}\label{eq:reciprocal_space_map}
    \bm{\widehat{R}}(\bm{r'}, \bm{q'}) = \left\| \sum_{l, m=0} \bm{a}_{l}^{0}(\bm{r'}, q') \bm{Y}_{l}^{0} \{ \Theta(\bm{r'}, q'), \Phi(\bm{r'}, q') \} \right\|^{2}.
\end{equation}

Equation \eqref{eq:reciprocal_space_map} only sums over the spherical harmonics functions $\bm{Y}_{l}^{0}$ and coefficients $\bm{a}_{l}^{0}(\bm{r'}, q')$ with $m=0$, since the nanostructures are assumed to have uniaxial symmetry.
As shown in Figure \ref{fig:spherical_harmonics}, the increasing $l$-value of the spherical harmonics functions $\bm{Y}_{l}^{0}$ corresponds to higher degree of uniaxiality.

\begin{figure}[h!]
    \centering
    \includesvg[height=0.4\textwidth]{../XRD_CT/Plotting/plots/sph_harm_samples.svg}
    \caption{Spherical harmonics functions $\bm{Y}_{l}^{0}$ with increasing $l$-values.}
    \label{fig:spherical_harmonics}
\end{figure}


Moreover, $q' = |\bm{q'}|$ is the magnitude of the reciprocal vector $\bm{q'}$, and $\Theta(\bm{r'}, q')$ and $\Phi(\bm{r'}, q')$ are the polar and azimuthal angles of the spherical harmonics function.
The angles may implicitly be defined by a series of coordinate transformations:

\begin{equation}\label{eq:coordinate_transformations}
    \begin{pmatrix}
        sin \Theta cos \Phi \\
        sin \Theta sin \Phi \\
        cos \Theta
    \end{pmatrix}
    = \bm{R}^{str} \bm{R}_{n}^{exp}
    \begin{pmatrix}
        sin \theta cos \phi \\
        sin \theta sin \phi \\
        cos \theta
    \end{pmatrix},
\end{equation}
\noindent
where $\bm{R}^{str}$ is the rotation of the zenith of the spherical harmonics function in each voxel. The preferred orientation is parameterised by $(\theta_{op}, \varphi_{op})$.
The input is the spherical position coordinates in the lab-coordinate system $\bm{r}$.
The rotation matrices in Equation \eqref{eq:coordinate_transformations} are defined as follows:
\begin{equation}
    \begin{split}
        \bm{R}^{str}(\bm{r'}) &=
        \begin{pmatrix}
            \costhetaop \cosphiop & \costhetaop \sinphiop & -\sinphiop  \\
            -\sinphiop            & \cosphiop             & 0           \\
            \sinthetaop \cosphiop & \sinthetaop \sinphiop & \costhetaop
        \end{pmatrix}\\
        \bm{R}_{n}^{exp} &=
        \begin{pmatrix}
            \cos \beta & \sin \alpha \cos \beta & -\cos\alpha \sin\beta \\
            0          & \cos \alpha            & \sin \alpha           \\
            \sin\beta  & - \sin\alpha \cos\beta & \cos\alpha \cos\beta
        \end{pmatrix}
    \end{split}
\end{equation}

Alternatively, one could model the form factor using a function where one parameter controls the amplitude, and the other controls the degree of uniaxiality, for instance
\begin{equation}\label{eq:exp_sin_squared}
    \bm{\widehat{R}}(\bm{r'}, \bm{q'}) = A^{2} \exp\{-B \sin^{2}\left( \Theta\right) \},
\end{equation}
with the resulting shape in threedimensional space shown in Figure \ref{fig:exp_sin_squared}.

\begin{figure}
    \centering
    \includesvg[height=0.4\textwidth]{../XRD_CT/Plotting/plots/exp_sin_samples.svg}
    \caption{The function in Equation \eqref{eq:exp_sin_squared} for some sets of parameters.}
    \label{fig:exp_sin_squared}
\end{figure}

There are a total of four parameters to optimise in Equation \eqref{eq:exp_sin_squared}: $A$, $B$, $\theta_{op}$, and $\varphi_{op}$.
In contrast, six possible parameters need to be optimised in Equation \eqref{eq:reciprocal_space_map}.
Moreover, the alternative representation provides with a continuous degree of uniaxiality
as opposed to the finite number of orders of spherical harmonics functions that can be included.
However, Equation \eqref{eq:reciprocal_space_map} is more general and can be used to model a linear combination of isotropic and uniaxial intensities.
The shape of the function for some sets of parameters is shown in Figure \ref{fig:exp_sin_squared}.


\section{Optimisation Algorithm}

SAXSTT is a Maximum Likelihood Estimation that can utilise gradient descent for optimisation, as mentioned in Chapter \ref{ch:optimisation}.
The forward pass of the tensor tomography is to calculate the resulting SAXS patterns from having the currently estimated model.
Then, the error between the calculated and measured SAXS patterns is calculated.
In terms of cost function expressions, there exist many options, but the expression used in Liebe et all \cite{liebi2018small} is:
\begin{equation}
    \epsilon_{q} = 2 \sum_{n, x, y, \phi} \omega_{n}(x,y,q,\phi) \left\{ \left( \widehat{I}_{n}(x,y,q,\phi) \right)^{\frac{1}{2}}  -  \left( \frac{ I_{n}(x,y,q,\phi) }{T_{n}(x,y)} \right)^{\frac{1}{2}} \right\}^{2},
\end{equation}
\noindent
where the respective parts of the expression are a binary mask $\omega$, the calculated SAXS patterns $\widehat{I}_{n}$, the measured SAXS patterns $I_{n}$, and the transmission intensity $T_{n}$.
The latter accounts for experimental absorption, which was another derived phenomenon, in addition to scattering, occurring when X-rays interact with matter,
as seen in Equation \eqref{eq:qm_interaction_Hamiltonian} in Chapter \ref{ch:scattering}.

The calculated SAXS patterns $\widehat{I}_{n}$ can be estimated in a number of ways.
However, it is generally an interpolation of the estimated reciprocal space map model $\bm{\widehat{R}}$ from the given projections.
Since the SAXS patterns are projections, it can be expressed as a sum over the beamline axis $z$,
\begin{equation}\label{eq:SAXS_intensity}
    \widehat{I}(x,y,q,\phi) = \sum_{z} \reciprocalspacemap.
\end{equation}

Additional effort is recommended to ensure that the Maximum Likelihood Estimation is converging towards the correct solution.
A separation of the optimisation into four steps was performed by Liebe \cite{liebi2018small} to ensure that the model was near convergence when optimising all parameters at once.
Therefore, the isotropic component $a_{0}$ is optimised first, followed by the preferred orientation $(\theta_{op}, \varphi_{op})$. In the second step, a guess of the uniaxial compontents has to be made.
Following this, the uniaxial components $a_{l}$ are optimised.
This allows the final optimisation to converge to a lower error value than if only one full optimisation were to be performed.


