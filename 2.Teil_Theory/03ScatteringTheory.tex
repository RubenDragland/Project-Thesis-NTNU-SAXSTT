\chapter{Scattering of X-rays}

\section{The Wave-Particle Duality and Differential Crossection}
% Big too long and tedious. Maybe just a short introduction to the wave-particle duality of x-rays.
As mentioned in section \ref{sec:CT_Xrays}, X-rays can be described as electromagnetic waves.
Furthermore, X-rays are described as plane waves, or Transverse electromagnetic waves (TEM-waves), assuming they are free, coherent, and monochromatic.
TEM-waves are characterised by a magnetic field $H$ and an electric field $E$ that are perpendicular to each other and to the direction of propagation, called the Poynting vector $S$.
%The mentioned fields and the Poynting vector may all be derived from the vector potential $A$.
Nevertheless, the expression of the TEM wave in terms of time $t$ and position $\bm{r}$ of the electric field $E$ is
\begin{equation}\label{eq:TEM_wave}
    \bm{E}(\bm{r},t) = \bm{\widehat{\epsilon}}  E_{0} \exp\left(-i\left(\omega t - \bm{k}\cdot\bm{r}\right)\right),
\end{equation}
where $\bm{\widehat{\epsilon}}$ is the unit vector in the direction of the electric field, $E_{0}$ is the amplitude of the electric field, $\omega$ is the angular frequency, and $\bm{k}$ is the wave vector.
Therefore can scattering events between X-rays and electrons, for instance, be described by
exertion of force between the electric field of the wave and the charge of the electron \cite{mcmorrow2011elements}.

However, quantisation of photons, which is performed within the scope of quantum mechanics, results in X-rays being described as photons.
This is a more useful description of X-rays in terms of scattering events, because it allows for particle-particle interactions between photons and electrons,
with transfer of momentum, elastic scattering, but also energy in events called inelastic scattering \cite{mcmorrow2011elements}.

The definition of the "differential scattering crossection" is also more intuitive from the perspective of quantisation.
The differential scattering crossection is the number of photons scattered relative to number of incoming photons per unit solid angle and time.
Mathematically, this is expressed in terms of scattering intensity $I_{s}$, the incoming flux $\Phi_{0}$, and the differential solid angle $\Delta\Omega$:

\begin{equation}\label{eq:scattering_crossection}
    \frac{dI_{s}}{d\Omega} = \frac{I_{s}}{\Phi_{0} \Delta\Omega}.
\end{equation}

\section{Classical Scattering Description}

In the classical description of scattering, the scattering vector $\bm{Q}$ and the atomic form factor $f(Q)$ are characteristic properties.

$\bm{Q}$ is linked to the phase shift of the scattering event, which can be understood from analysing equation \ref{eq:TEM_wave}. It is defined as
\begin{equation}\label{eq:scattering_vector}
    \bm{Q} = \bm{k} - \bm{k'},
\end{equation}
where $\bm{k}$ is the incoming wave vector and $\bm{k'}$ is the scattered wave vector.

The atomic form factor $f(Q)$ is a function of the scattering vector $\bm{Q}$, from Equation \eqref{eq:scattering_vector},
and describes the scattering of X-rays by the electron density of the atom.
Generally, it is a Fourier transform of the electron density distribution of the atom $\rho(\bm{r})$,
\begin{equation}
    f(Q) = \int \rho(\bm{r}) \exp\left(i\bm{Q}\cdot\bm{r}\right) d\bm{r}.
\end{equation}

As a result,
%Scattering Intensity

%Anything else?
%This is going poorly...





\section{Time Dependent Pertubation Theory} %Which to describe?

\section{Reciprocal Space Related to Electron Density}
