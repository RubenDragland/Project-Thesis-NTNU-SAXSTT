\chapter{Introduction}


Machine learning has grown to be a powerful tool which is applied to many aspects of society,
either through marketing of commercial products, or in the development of new technologies aimed at easing the lives of people.
However, there is also an ongoing machine learning revolution in the field of physics,
where machine learning is used to solve complex optimisation tasks, thereby revealing new insights that were previously nonextractable.
One instance where machine learning has been proven to be powerful is in the field of computed tomography (CT) and related algorithms thereof.
CT is known as a medical imaging technique that uses X-rays to produce a 3D model of the internal objects of the patient that was scanned.
The technique is also highly applicable to materials science, where micro-CT ($\mu CT$) is being used to image the internal structure of microporous materials.
However, the term computed tomography refers actually to a broader class of algorithms where 3D models are reconstructed from 2D projections.
One of these algorithms is called Small Angle X-ray Scattering Tensor Tomography (SAXSTT), which distinguishes itself from conventional CT by the fact that the technique
reconstructs a 3D vector field based on X-ray scattering measurements. Conventional CT reconstructs, as mentioned, a 3D scalar field from X-ray absorption measurements.
SAXSTT is therefore able to do harmless and efficient orientation mapping of uniaxial nanostructures throughout any three-dimensional amorphous sample with micrometer spatial resolution.
The technique combines the power of gradient descent with the quantum mechanical relations between scattering pattern and real-space orientation.

In this thesis, optimisation of Small Angle Scattering X-ray Tensor Tomography is investigated.
The algorithm is still young and has not yet been developed to its full potential.
One important tool in the continued development of this algorithm is automatic differentiation.
Therefore, the first step in this project work is to implement the gradient calculation of the SAXSTT algorithm using automatic differentiation.
With this in place, the reconstruction technique can be optimised and expanded in a versatile and efficient manner.
An initial demonstration of such optimisation is also presented in this thesis, by testing an alternative functional to model the 3D reciprocal space map.